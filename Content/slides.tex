\begin{frame}{Introduction}
	The Capacitated Vehicle Routing Problem (\textbf{CVRP}) \parencite{dantzig1959}
	is an \textcolor{blue}{NP-hard combinatorial optimization routing problem} with applications in \textcolor{blue}{logistics}, such as delivery of goods or services to customers.

	\vspace{0.3cm}

	\begin{columns}
		\begin{column}{.35\textwidth}
			\centering
			\includegraphics[height=4cm]{Imgs/CVRP-example.cropped.pdf}
		\end{column}
		\begin{column}{.65\textwidth}
			CVRP defined on an undirected graph. We are given:
			\begin{itemize}
				\item Node's \textcolor{blue}{location} within a road network.
				\item The location of the \textcolor{blue}{depot node}.
				\item The \textcolor{blue}{demand} of each customer.
				\item Number of \textcolor{blue}{available vehicles} with their \textcolor{blue}{capacity}.
			\end{itemize}
			Objective:
			\begin{itemize}
				\item \textcolor{red}{Find each vehicle's route minimizing the overall routing cost while serving \textbf{all} customers}.
			\end{itemize}
		\end{column}
	\end{columns}

\end{frame}

\begin{frame}{Branch-price-and-cut}
\end{frame}

\begin{frame}{The Pricing Sub-problem}
	To advance the column generation, the \textcolor{blue}{pricer}, a critical component in BPC frameworks, needs to solve the \textcolor{blue}{pricing sub-problem} (PP):
	\begin{itemize}
		\item An \textcolor{blue}{Elementary Shortest Path Problem with Capacity Constraints} (\textbf{ESPPCC}) in a reduced cost network with negative cycles.
		      \begin{itemize}
			      \item NP-hard problem \parencite{dror1994}.
		      \end{itemize}
		\item \textbf{Relax elementarity condition} to make it solvable in pseudo-polynomial time:
		      \begin{itemize}
			      \item $q$-routes with 2-cycles elimination \parencite{christofides1969}.
			      \item $q$-routes with arbitrary $k$-cycles elimination \parencite{christofides1969}.
			      \item ng-routes \parencite{baldacci2011}.
		      \end{itemize}
		\item State-of-the-art solutions for the \textcolor{red}{relaxed} PP are based on \textcolor{blue}{dynamic programming}:
		      \begin{itemize}
			      \item \textbf{labeling algorithm} \parencite{desrochers1992, feillet2004}.
		      \end{itemize}
	\end{itemize}
\end{frame}

\begin{frame}{Thesis Contributions}
\end{frame}

\begin{frame}{Implementation}
\end{frame}

\begin{frame}{Empirical evaluation}
	\begin{itemize}
		\item Running time measured through \textbf{performance profiles} \parencite{dolan2002}.
	\end{itemize}
\end{frame}

\begin{frame}{Results (1/2)}

\end{frame}

\begin{frame}{Results (2/2)}

\end{frame}

\begin{frame}{Conclusions and Future Work}
	\cite{jepsen2014}
\end{frame}

\begin{frame}{The end}
	\begin{center}
		\begingroup
		\fontsize{18pt}{18pt}\selectfont
		Thank, you.
		\endgroup
	\end{center}
\end{frame}

\appendix

\begin{frame}
\end{frame}

\begin{frame}
	\begin{center}
		\begingroup
		\fontsize{18pt}{18pt}\selectfont
		Appendix.
		\endgroup
	\end{center}
\end{frame}

\begin{frame}{Integer Programming}
	MIP solvers are rather general and can be used to solve a wide range of problems from various fields \parencite{bixby2007progress}.
	MIP models are, in spirit, a way to mathematically program a solver to achieve the desired solution.
	A MIP solver can solve a mixed-integer linear programming formulation
	expressed as \parencite{wolsey1999integer}:
	\begin{align}
		 & \max_{x, y} & c^T x + d^T y                                 \\
		 & \text{s.t.} & A x + B y \le b  \label{eq:general-mip-bound} \\
		 &             & x \in \R^n                                    \\
		 &             & y \in \Z_{+}^k,
	\end{align}
	where $A \in \R^{m \times n}, B \in \R^{m \times k}$ are matrices and
	$c \in \R^n, d \in R^k, b \in \R^m$ are vector coefficients.
	The bound in \cref{eq:general-mip-bound} can also be rewritten in equality and/or greater form.

\end{frame}
